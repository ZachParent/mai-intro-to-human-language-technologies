\documentclass{beamer}

% Theme and color settings
\usetheme{CambridgeUS}
\usecolortheme{default}

% Package imports
\usepackage{graphicx}
\usepackage{booktabs}
\usepackage{url}

% Title information
\title{Semantic Text Similarity:\\A Feature Importance Analysis}
\author{Your Name}
\institute{Universitat Politècnica de Catalunya}
\date{\today}

\begin{document}

% Title frame
\begin{frame}
    \titlepage
\end{frame}

% Table of contents
\begin{frame}{Outline}
    \tableofcontents
\end{frame}

% Introduction
\section{Introduction}
\begin{frame}{Introduction}
    \begin{itemize}
        \item Semantic Text Similarity (STS) is crucial for many NLP tasks
        \item Challenge: Which features best capture semantic similarity?
        \item Our approach: Unbiased feature analysis using Random Forests
    \end{itemize}
\end{frame}

% Approach
\section{Approach}
\begin{frame}{Methodology}
    \begin{itemize}
        \item Generated ~2000 potential features
        \item Used Random Forest's feature importance capabilities
        \item Let the data guide feature selection
    \end{itemize}
\end{frame}

% Results
\section{Results}
\begin{frame}{Top Features}
    \begin{itemize}
        \item Jaccard similarity dominates (7 of top 10)
        \item Common pipeline steps: lemmatization, stopwords, n-grams
        \item Top feature accounts for 20\% importance
    \end{itemize}
\end{frame}

% Conclusions
\section{Conclusions}
\begin{frame}{Conclusions}
    \begin{itemize}
        \item Simple features can be highly effective
        \item Pipeline complexity isn't always better
        \item Character-level analysis with n-grams shows promise
    \end{itemize}
\end{frame}

\end{document}
